\documentclass{article}

\usepackage[utf8]{inputenc}
\usepackage[french]{babel}
\usepackage{amsmath}
\usepackage{amssymb}
\usepackage[left=2cm,right=2cm,top=2cm,bottom=2cm]{geometry}

\newcommand{\PP}[1]{\mathbb{P}\left[ #1 \right ]}
\newcommand{\EE}[1]{\mathbb{E}\left[ #1 \right ]}
\newcommand{\N}{\mathbb{N}}
\newcommand{\M}{\mathcal{M}}
\newcommand{\oCPO}{$\omega$-CPO }
\newcommand{\Cr}[1]{[\![#1]\!]}
\newcommand{\Co}[1]{\left \langle #1 \right \rangle}
\title{APVS -- DM 3}
\author{Charrondière Raphaël}
\date{24 Avril 2015}
\begin{document}
\maketitle
\section*{Warmup on CPOs}
\subsection*{Question 1}
$A$, $B$ sont des \oCPO s

$A\times B$ en est aussi : on définit $\leq_{A\times B}$ l'ordre naturel sur les produits.

$\bot_{A\times B}=(\bot_A,\bot_B)$ : par définition $\forall a\in A,b\in B\ (\bot_A,\bot_B)\leq (a,b)$ car $a\leq\bot_A$ and $v\leq\bot_B$

Prenons une suite croissante $(a_n,b_n)_n \in (A\times B)^\mathbb{N}$. Par définition de l'ordre $(a_n)_n, (b_n)_n$ sont aussi croissantes, donc ont un sup, notons les $a,b$. Montrons $(a,b)$ est le sup de $(a_n,b_n)_n$. De un, il est plus grand que tout les termes de la suite. Prenons le sup $(a',b')$ $a'$ est plus grand que tout les $a_n$, de même pour b. Donc soit $a'\leq a$, soit $a_n$ n'a pas de sup, ce qui est impossible car A est un $\omega-CPOS$. Idem pour  b. Donc $(a,b)\leq (a',b')$

Pour l'\oCPO 1 on prend le singleton {1} munit de la relation $1\leq 1$. Trivialement toute fonction $A \rightarrow 1$ est Scott-continue et unique.

\subsection*{Question 2}
On définit $\circ : [Y \rightarrow Z_\bot] \times [X \rightarrow Y_\bot] \rightarrow [X \rightarrow Z_\bot]$ par 
$$\forall x \in X,  f\circ g(x) = \bot \text{ si } g(x) = \bot $$  $$f\circ g(x) =f(g(x)) \text{ sinon}$$

On notera $\circ$ comme un opérateur binaire

\begin{itemize}
    \item Elle est monotone. Si $(f, g) \leq (f', g')$ deux cas :
    \begin{itemize}
        \item si $f(x) = \bot$ alors $f\circ g(x) = \bot \leq f'\circ g'(x)$ (car $f'\circ g'(x) \in Z$) 
        \item si $f(x) \neq \bot$ $f\circ g(x) = f(g(x)) \leq f'(g'(x)) = f\circ g(x)$ (car $f'$ est monotone et comme $g' \geq g$, $g'(x) \geq g(x) > \bot$ donc $g'(x) \neq \bot$).
    \end{itemize}
    \item Soit $(f_n, g_n)_{n \in \N}$ une suite croissante. On pose $(f, g) = \sup (f_n, g_n)$. Soit $x \in X$ quelconque. Deux cas :
    \begin{itemize}
        \item $\forall n \: g_n(x) = \bot$ alors $g(x) = \bot$ par continuité et $\circ(\sup (f_n, g_n))(x) = f\circ g(x) = \bot = \sup f_n\circ g_n(x)$.
        \item $\exists m \: g_m(x) \neq \bot$ alors par croissance $\forall n \geq m \: g_n(x) \neq \bot$ donc $g(x) \neq \bot$ et $\circ(\sup (f_n, g_n))(x) = f\circ g(x) = f(g(x)) = \sup_{n \geq m} f_n(g_n(x)) = \sup f_n\circ g_n(x)$.
    \end{itemize}
\end{itemize}

\subsection*{Question 3}

Pour répondre à $f(F(f))=F(f)$, $F$ cherche un point fixe de $f$. 

Prenons $F:f\in [A \rightarrow A] \rightarrow \sup_n f^n(\bot_A) \in A$

A est un \oCPO donc F est bien définie.

Vérifions que $F$ soit Scott-continue.

$F$ est croissante : si $f\leq g$ $\forall n \in \N f^n(\bot_A)\leq g^n(\bot_A)$ (par définition de l'ordre sur les fonctions) donc $\sup_n f^n(\bot_A) \leq \sup_n g^n(\bot_A)$ c'est à dire $F(f)\leq F(g)$

Soit $f_n$ une suite croissante $F(\sup f_n)=\sup_k \sup_n f_n^k(\bot_A)=\sup_n \sup_k f_n^k(\bot_A)=\sup_n F(f_n)$


\subsection*{Question 4}

On utilise ici la fonction $Cond : (\M \rightarrow \{ true, false \}) \times [\M \rightarrow \M_\bot] \times [\M \rightarrow \M_\bot] \rightarrow [\M \rightarrow \M_\bot]$ définie en cours qui est continue.

\begin{itemize}
    \item $C = \texttt{skip}$. 
    
    Alors on prend $\Cr{C} = Id_\M \in [\M \rightarrow \M_\bot]$. On a bien $\Co{C, \sigma} \Downarrow \sigma'$  si et seulement si  $\Cr{C}(\sigma) = \sigma'$.
    
    \item $C = C' ; C''$. 
    
    On définit $\Cr{C}= \Cr{C''} \circ \Cr{C'}$ (Cf question 2).
    
      On a $\Cr{C} \in [\M \rightarrow \M_\bot]$ et $\Co{C, \sigma} \Downarrow \sigma' \Leftrightarrow\exists \sigma''$ $\Co{C', \sigma} \Downarrow \sigma''$ et $\Co{C'', \sigma''} \Downarrow \sigma'\Leftrightarrow\Cr{C''}(\Cr{C'}(\sigma)) = \sigma'$ 
      
      Donc d'après la règle d'inférence $\Cr{C}=\Cr{C' ; C''}$
    
    \item $C = \texttt{r := a}$ 
    
    On définit $\Cr{C} = \sigma \mapsto \sigma\Co{\mathcal{A}[e]\sigma/r}$.
    
    \item $C = \texttt{if } b \texttt{ then } C_1 \texttt{ else } C_2$. 
    
    On définit $\Cr{C} = Cond(\mathcal{B}(b), C_1, C_2) \in [\M \rightarrow \M_\bot]$ qui est valide par définition de $Cond$.
    
    \item $C = \texttt{while } b \texttt{ do } C'$. 
    
    On définit $\Cr{C} = F(f \mapsto Cond(\mathcal{B}(b),  f\circ \Cr{C'}, Id_\M))$ (Cf question 3) qui est valide car $C'$ est ré-exécutée  tant que b est vrai.
    
      $f \mapsto Cond(\mathcal{B}(b), C' \circ f, Id_\M)$ est continue comme composé de fonctions continue (Cf question 2). Donc $\Cr{C}$ est bien définie (et continue).
\end{itemize}

\subsection*{Question 5}

Notation : Soit $I \in T(A)$ quelconque. On note alors $I_1 : \M_\bot \rightarrow A$ et $I_2 : \M_\bot \rightarrow \M_\bot$ les deux projections naturelles associées à $I$.

On pose alors $T(f)(I) = \sigma \mapsto (f(I_1(\sigma)), I_2(\sigma))$. Naturellement $T\in (A \rightarrow B) \rightarrow (T(A)\rightarrow T(B))$

$\forall I\in T(A), \ T(Id_A)(I)=(Id_A(I_1),I_2)=(I_1,I_2)=I$ donc $T(Id_A)(I)=Id_{T(A)}$.

Soient $f,g$ $\forall I\in T(A),\ T(g\circ f)(I)=(g\circ f(I_1),I_2)=T(g)(f(I_1),I_2)=T(g)\circ T(f)(I)$

Enfin $T(f)$ est continue : on a vu que l'on conservait la continuité par composition, couplage. Considérons que l'ordre sur les couples est l'ordre produit, ce qui permet d'affirmer que la continuité est conservée par projection. Ainsi T est continue.


\subsection*{Question 6}

On pose $\eta_A(a) = \sigma \mapsto (a, \sigma)$ et $\mu_A(I) = \sigma \mapsto I_1(\sigma)(I_2(\sigma))$. 

$\eta_A$ est trivialement continue et $\mu_A$ l'est aussi comme composition de fonctions continues ($I_1$ et $I_2$ le sont).

\begin{itemize}
\item Soit $I \in T(A)$ et $\sigma \in \M$.

  On a $(\mu_A \circ T(\eta_A))(I)(\sigma) = (\mu_A(T(\eta_A)(I)))(\sigma) = (\mu_A(\sigma' \mapsto (\eta_A(I_1(\sigma')), I_2(\sigma')))(\sigma)$
  
   $ = (\sigma' \mapsto (\eta_A(I_1(\sigma'))))(\sigma)((\sigma' \mapsto I_2(\sigma'))(\sigma)) = \eta_A (I_1(\sigma))(I_2(\sigma)) = (I_1(\sigma), I_2(\sigma)) = I(\sigma)$.

De même on a $(\mu_A \circ \eta_{T(A)})(I)(\sigma) = (\mu_A(\eta_{T(A)}(I)))(\sigma) = (\mu_A(\sigma' \mapsto (I, \sigma')))(\sigma) = (\sigma' \mapsto I)(\sigma)('\sigma' \mapsto \sigma')(\sigma)) = I(\sigma)$.

Ainsi on a bien $\mu_A \circ T(\eta_A) = \mu_A \circ \eta_{T(A)} = Id_{T(A)}$.

\item Soit $I \in T(T(T(A)))$.

 On a $(\mu_A \circ \mu_{T(A)})(I) = \mu_A(\mu_{T(A)}(I)) = \mu_A(\sigma' \mapsto I_1(\sigma')(I_2(\sigma'))) = \sigma \mapsto {I_1}_1(\sigma)(I_2(\sigma))({I_1}_2(\sigma)(I_2(\sigma)))$

De plus $(\mu_A \circ T(\mu_A))(I) = \mu_A(T(\mu_A)(I)) = \mu_A(\sigma' \mapsto (\mu_A(I_1(\sigma')), I_2(\sigma'))) = \sigma \mapsto (\mu_A(I_1(\sigma)))(I_2(\sigma)) = \sigma \mapsto (\sigma' \mapsto {I_1}_1(\sigma)(\sigma')({I_1}_2(\sigma)(\sigma')))(I_2(\sigma)) = \sigma \mapsto {I_1}_1(\sigma)(I_2(\sigma))({I_1}_2(\sigma)(I_2(\sigma))))$
 
 Donc on a bien $\mu_A \circ \mu_{T(A)} = \mu_A \circ T(\mu_A)$.
 
 \end{itemize}
\subsection*{Question 7}
\subsection*{Question 8}
\subsection*{Question 9}
\subsection*{Question 10}
\subsection*{Question 11}
\end{document}
