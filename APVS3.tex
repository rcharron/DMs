\documentclass{article}

\usepackage[utf8]{inputenc}
\usepackage[french]{babel}
\usepackage{amsmath}
\usepackage{amssymb}
\usepackage[left=2cm,right=2cm,top=2cm,bottom=2cm]{geometry}

\newcommand{\PP}[1]{\mathbb{P}\left[ #1 \right ]}
\newcommand{\EE}[1]{\mathbb{E}\left[ #1 \right ]}
\newcommand{\N}{\mathbb{N}}
\newcommand{\oCPO}{$\omega$-CPO }

\title{APVS -- DM 3}
\author{Charrondière Raphaël}
\date{24 Avril 2015}
\begin{document}
\maketitle
\section*{Warmup on CPOs}
\subsection*{Question 1}
$A$, $B$ sont des \oCPO s

$A\times B$ en est aussi : on définit $\leq_{A\times B}$ l'ordre naturel sur les produits.

$\bot_{A\times B}=(\bot_A,\bot_B)$ : par définition $\forall a\in A,b\in B\ (\bot_A,\bot_B)\leq (a,b)$ car $a\leq\bot_A$ and $v\leq\bot_B$

Prenons une suite croissante $(a_n,b_n)_n \in (A\times B)^\mathbb{N}$. Par définition de l'ordre $(a_n)_n, (b_n)_n$ sont aussi croissantes, donc ont un sup, notons les $a,b$. Montrons $(a,b)$ est le sup de $(a_n,b_n)_n$. De un, il est plus grand que tout les termes de la suite. Prenons le sup $(a',b')$ $a'$ est plus grand que tout les $a_n$, de même pour b. Donc soit $a'\leq a$, soit $a_n$ n'a pas de sup, ce qui est impossible car A est un $\omega-CPOS$. Idem pour  b. Donc $(a,b)\leq (a',b')$

Pour l'\oCPO 1 on prend le singleton {1} munit de la relation $1\leq 1$. Trivialement toute fonction $A \rightarrow 1$ est Scott-continue et unique.

\subsection*{Question 2}
On définit $\circ : [Y \rightarrow Z_\bot] \times [X \rightarrow Y_\bot] \rightarrow [X \rightarrow Z_\bot]$ par 
$$\forall x \in X,  f\circ g(x) = \bot \text{ si } g(x) = \bot $$  $$f\circ g(x) =f(g(x)) \text{ sinon}$$

On notera $\circ$ comme un opérateur binaire

\begin{itemize}
    \item Elle est monotone. Si $(f, g) \leq (f', g')$ deux cas :
    \begin{itemize}
        \item si $f(x) = \bot$ alors $f\circ g(x) = \bot \leq f'\circ g'(x)$ (car $f'\circ g'(x) \in Z$) 
        \item si $f(x) \neq \bot$ $f\circ g(x) = f(g(x)) \leq f'(g'(x)) = f\circ g(x)$ (car $f'$ est monotone et comme $g' \geq g$, $g'(x) \geq g(x) > \bot$ donc $g'(x) \neq \bot$).
    \end{itemize}
    \item Soit $(f_n, g_n)_{n \in \N}$ une suite croissante. On pose $(f, g) = \sup (f_n, g_n)$. Soit $x \in X$ quelconque. Deux cas :
    \begin{itemize}
        \item $\forall n \: g_n(x) = \bot$ alors $g(x) = \bot$ par continuité et $\circ(\sup (f_n, g_n))(x) = f\circ g(x) = \bot = \sup f_n\circ g_n(x)$.
        \item $\exists m \: g_m(x) \neq \bot$ alors par croissance $\forall n \geq m \: g_n(x) \neq \bot$ donc $g(x) \neq \bot$ et $\circ(\sup (f_n, g_n))(x) = f\circ g(x) = f(g(x)) = \sup_{n \geq m} f_n(g_n(x)) = \sup f_n\circ g_n(x)$.
    \end{itemize}
\end{itemize}

\subsection*{Question 3}

Pour répondre à $f(F(f))=F(f)$, $F$ cherche un point fixe de $f$. 

Prenons $F:f\in [A \rightarrow A] \rightarrow \sup_n f^n(\bot_A) \in A$

A est un \oCPO donc F est bien définie.

Vérifions que $F$ soit Scott-continue.

$F$ est croissante : si $f\leq g$ $\forall n \in \N f^n(\bot_A)\leq g^n(\bot_A)$ (par définition de l'ordre sur les fonctions) donc $\sup_n f^n(\bot_A) \leq \sup_n g^n(\bot_A)$ c'est à dire $F(f)\leq F(g)$

Soit $f_n$ une suite croissante $F(\sup f_n)=\sup_k \sup_n f_n^k(\bot_A)=\sup_n \sup_k f_n^k(\bot_A)=\sup_n F(f_n)$
\subsection*{Question 4}
\subsection*{Question 5}
\subsection*{Question 6}
\subsection*{Question 7}
\subsection*{Question 8}
\subsection*{Question 9}
\subsection*{Question 10}
\subsection*{Question 11}
\end{document}
