\documentclass{article}

\usepackage[utf8]{inputenc}
\usepackage[french]{babel}
\usepackage{amsmath}
\usepackage{amssymb}
\usepackage[left=2cm,right=2cm,top=2cm,bottom=2cm]{geometry}

\newcommand{\PP}[1]{\mathbb{P}\left[ #1 \right ]}
\newcommand{\EE}[1]{\mathbb{E}\left[ #1 \right ]}
\newcommand{\N}{\mathbb{N}}
\newcommand{\M}{\mathcal{M}}
\newcommand{\oCPO}{$\omega$-CPO }
\newcommand{\Cr}[1]{[\![#1]\!]}
\newcommand{\Co}[1]{\left \langle #1 \right \rangle}
\newcommand{\comp}{\text{comp}}
\newcommand{\bool}{\mathcal{B}\textit{ool}}
\title{APVS -- DM 3}
\author{Charrondière Raphaël}
\date{24 Avril 2015}
\begin{document}
\maketitle
\section*{Warmup on CPOs}
\subsection*{Question 1}
$A$, $B$ sont des \oCPO s

$A\times B$ en est aussi : on définit $\leq_{A\times B}$ l'ordre naturel sur les produits.

$\bot_{A\times B}=(\bot_A,\bot_B)$ : par définition $\forall a\in A,b\in B\ (\bot_A,\bot_B)\leq (a,b)$ car $a\leq\bot_A$ and $v\leq\bot_B$

Prenons une suite croissante $(a_n,b_n)_n \in (A\times B)^\mathbb{N}$. Par définition de l'ordre $(a_n)_n, (b_n)_n$ sont aussi croissantes, donc ont un sup, notons les $a,b$. Montrons $(a,b)$ est le sup de $(a_n,b_n)_n$. De un, il est plus grand que tout les termes de la suite. Prenons le sup $(a',b')$ $a'$ est plus grand que tout les $a_n$, de même pour b. Donc soit $a'\leq a$, soit $a_n$ n'a pas de sup, ce qui est impossible car A est un $\omega-CPOS$. Idem pour  b. Donc $(a,b)\leq (a',b')$

Pour l'\oCPO 1 on prend le singleton {1} munit de la relation $1\leq 1$. Trivialement toute fonction $A \rightarrow 1$ est Scott-continue et unique.

\subsection*{Question 2}
On définit $\circ : [Y \rightarrow Z_\bot] \times [X \rightarrow Y_\bot] \rightarrow [X \rightarrow Z_\bot]$ par 
$$\forall x \in X,  f\circ g(x) = \bot \text{ si } g(x) = \bot $$  $$f\circ g(x) =f(g(x)) \text{ sinon}$$

On notera $\circ$ comme un opérateur binaire

\begin{itemize}
    \item Elle est monotone. Si $(f, g) \leq (f', g')$ deux cas :
    \begin{itemize}
        \item si $f(x) = \bot$ alors $f\circ g(x) = \bot \leq f'\circ g'(x)$ (car $f'\circ g'(x) \in Z$) 
        \item si $f(x) \neq \bot$ $f\circ g(x) = f(g(x)) \leq f'(g'(x)) = f\circ g(x)$ (car $f'$ est monotone et comme $g' \geq g$, $g'(x) \geq g(x) > \bot$ donc $g'(x) \neq \bot$).
    \end{itemize}
    \item Soit $(f_n, g_n)_{n \in \N}$ une suite croissante. On pose $(f, g) = \sup (f_n, g_n)$. Soit $x \in X$ quelconque. Deux cas :
    \begin{itemize}
        \item $\forall n \: g_n(x) = \bot$ alors $g(x) = \bot$ par continuité et $\circ(\sup (f_n, g_n))(x) = f\circ g(x) = \bot = \sup f_n\circ g_n(x)$.
        \item $\exists m \: g_m(x) \neq \bot$ alors par croissance $\forall n \geq m \: g_n(x) \neq \bot$ donc $g(x) \neq \bot$ et $\circ(\sup (f_n, g_n))(x) = f\circ g(x) = f(g(x)) = \sup_{n \geq m} f_n(g_n(x)) = \sup f_n\circ g_n(x)$.
    \end{itemize}
\end{itemize}

\subsection*{Question 3}

Pour répondre à $f(F(f))=F(f)$, $F$ cherche un point fixe de $f$. 

Prenons $F:f\in [A \rightarrow A] \rightarrow \sup_n f^n(\bot_A) \in A$

A est un \oCPO donc F est bien définie.

Vérifions que $F$ soit Scott-continue.

$F$ est croissante : si $f\leq g$ $\forall n \in \N f^n(\bot_A)\leq g^n(\bot_A)$ (par définition de l'ordre sur les fonctions) donc $\sup_n f^n(\bot_A) \leq \sup_n g^n(\bot_A)$ c'est à dire $F(f)\leq F(g)$

Soit $f_n$ une suite croissante $F(\sup f_n)=\sup_k \sup_n f_n^k(\bot_A)=\sup_n \sup_k f_n^k(\bot_A)=\sup_n F(f_n)$


\subsection*{Question 4}

On utilise ici la fonction $Cond : (\M \rightarrow \{ true, false \}) \times [\M \rightarrow \M_\bot] \times [\M \rightarrow \M_\bot] \rightarrow [\M \rightarrow \M_\bot]$ définie en cours qui est continue.

\begin{itemize}
    \item $C = \texttt{skip}$
    
   On définit $\Cr{C} = Id_\M \in [\M \rightarrow \M_\bot]$. On a $\Co{C, \sigma} \Downarrow \sigma'$  si et seulement si  $\Cr{C}(\sigma) = \sigma'$.
    
    \item $C = C' ; C''$
    
    On définit $\Cr{C}= \Cr{C''} \circ \Cr{C'}$ (Cf question 2).
    
      On a $\Cr{C} \in [\M \rightarrow \M_\bot]$ et $\Co{C, \sigma} \Downarrow \sigma' \Leftrightarrow\exists \sigma''$ $\Co{C', \sigma} \Downarrow \sigma''$ et $\Co{C'', \sigma''} \Downarrow \sigma'\Leftrightarrow\Cr{C''}(\Cr{C'}(\sigma)) = \sigma'$ 
      
      Donc d'après la règle d'inférence $\Cr{C}=\Cr{C' ; C''}$
    
    \item $C = \texttt{r := a}$ 
    
    On définit $\Cr{C} = \sigma \mapsto \sigma\Co{\mathcal{A}[e]\sigma/r}$.
    
    \item $C = \texttt{if } b \texttt{ then } C_1 \texttt{ else } C_2$
    
    On définit $\Cr{C} = Cond(\mathcal{B}(b), C_1, C_2) \in [\M \rightarrow \M_\bot]$ qui est valide par définition de $Cond$.
    
    \item $C = \texttt{while } b \texttt{ do } C'$
    
    On définit $\Cr{C} = F(f \mapsto Cond(\mathcal{B}(b),  f\circ \Cr{C'}, Id_\M))$ (Cf question 3) qui est valide car $C'$ est ré-exécutée  tant que b est vrai.
    
      $f \mapsto Cond(\mathcal{B}(b), C' \circ f, Id_\M)$ est continue comme composé de fonctions continue (Cf question 2). Donc $\Cr{C}$ est bien définie (et continue).
\end{itemize}

\subsection*{Question 5}

Notation : Soit $I \in T\left (A\right )$ quelconque. On note alors $I_1 : \M_\bot \rightarrow A$ et $I_2 : \M_\bot \rightarrow \M_\bot$ les deux projections naturelles associées à $I$.

On pose alors $T\left (f\right )\left (I\right ) = \sigma \mapsto \left (f\left (I_1\left (\sigma\right )\right ), I_2\left (\sigma\right )\right )$. Naturellement $T\in \left (A \rightarrow B\right ) \rightarrow \left (T\left (A\right )\rightarrow T\left (B\right )\right )$

$\forall I\in T\left (A\right ), \ T\left (Id_A\right )\left (I\right )=\left (Id_A\left (I_1\right ),I_2\right )=\left (I_1,I_2\right )=I$ donc $T\left (Id_A\right )\left (I\right )=Id_{T\left (A\right )}$.

Soient $f,g$ $\forall I\in T\left (A\right ),\ T\left (g\circ f\right )\left (I\right )=\left (g\circ f\left (I_1\right ),I_2\right )=T\left (g\right )\left (f\left (I_1\right ),I_2\right )=T\left (g\right )\circ T\left (f\right )\left (I\right )$

Enfin $T\left (f\right )$ est continue : on a vu que l'on conservait la continuité par composition, couplage. Considérons que l'ordre sur les couples est l'ordre produit, ce qui permet d'affirmer que la continuité est conservée par projection. Ainsi T est continue.


\subsection*{Question 6}

On pose $\eta_A\left (a\right ) = \sigma \mapsto \left (a, \sigma\right )$ et $\mu_A\left (I\right ) = \sigma \mapsto I_1\left (\sigma\right )\left (I_2\left (\sigma\right )\right )$. 

$\eta_A$ est trivialement continue et $\mu_A$ l'est aussi comme composition de fonctions continues  ($I_1$ et $I_2$ le sont ).

\begin{itemize}
\item Soit $I \in T\left (A\right )$ et $\sigma \in \M$.

  On a $\left (\mu_A \circ T\left (\eta_A\right )\right )\left (I\right )\left (\sigma\right ) = \left (\mu_A\left (T\left (\eta_A\right )\left (I\right )\right )\right )\left (\sigma\right ) = \left (\mu_A\left (\sigma' \mapsto \left (\eta_A\left (I_1\left (\sigma'\right )\right ), I_2\left (\sigma'\right )\right )\right )\left (\sigma \right )\right )$
  
   $ = \left (\sigma' \mapsto \left (\eta_A\left (I_1\left (\sigma'\right )\right )\right )\right )\left (\sigma\right )\left (\left (\sigma' \mapsto I_2\left (\sigma'\right )\right )\left (\sigma\right )\right ) = \eta_A \left (I_1\left (\sigma\right )\right )\left (I_2\left (\sigma\right )\right ) = \left (I_1\left (\sigma\right ), I_2\left (\sigma\right )\right ) = I\left (\sigma\right )$.

De même on a $\left (\mu_A \circ \eta_{T\left (A\right )}\right )\left (I\right )\left (\sigma\right ) = \left (\mu_A\left (\eta_{T\left (A\right )}\left (I\right )\right )\right )\left (\sigma\right ) = \left (\mu_A\left (\sigma' \mapsto \left (I, \sigma'\right )\right )\right )\left (\sigma\right ) = \left (\sigma' \mapsto I\right )\left (\sigma\right )\left (\left (\sigma' \mapsto \sigma'\right )\left (\sigma\right )\right ) = I\left (\sigma\right )$.

Ainsi on a bien $\mu_A \circ T\left (\eta_A\right ) = \mu_A \circ \eta_{T\left (A\right )} = Id_{T\left (A\right )}$.

\item Soit $I \in T\left (T\left (T\left (A\right )\right )\right )$.

 On a $\left (\mu_A \circ \mu_{T\left (A\right )}\right )\left (I\right ) = \mu_A\left (\mu_{T\left (A\right )}\left (I\right )\right ) = \mu_A\left (\sigma' \mapsto I_1\left (\sigma'\right )\left (I_2\left (\sigma'\right )\right )\right ) = \sigma \mapsto {I_1}_1\left (\sigma\right )\left (I_2\left (\sigma\right )\right )\left ({I_1}_2\left (\sigma\right )\left (I_2\left (\sigma\right )\right )\right )$

De plus $\left (\mu_A \circ T\left (\mu_A\right )\right )\left (I\right ) = \mu_A\left (T\left (\mu_A\right )\left (I\right )\right ) = \mu_A\left (\sigma' \mapsto \left (\mu_A\left (I_1\left (\sigma'\right )\right ), I_2\left (\sigma'\right )\right )\right ) =$ 
$ \sigma \mapsto \left (\mu_A\left (I_1\left (\sigma\right )\right )\right )\left (I_2\left (\sigma\right )\right ) = \sigma \mapsto \left (\sigma' \mapsto {I_1}_1\left (\sigma\right )\left (\sigma'\right )\left ({I_1}_2\left (\sigma\right )\left (\sigma'\right )\right )\right )\left (I_2\left (\sigma\right )\right ) $
$= \sigma \mapsto {I_1}_1\left (\sigma\right )\left (I_2\left (\sigma\right )\right )\left ({I_1}_2\left (\sigma\right )\left (I_2\left (\sigma\right )\right )\right )$
 
 Donc on a bien $\mu_A \circ \mu_{T\left (A\right )} = \mu_A \circ T\left (\mu_A\right )$.
 
 \end{itemize}
 
\subsection*{Question 7}

Posons $s:\left (I, b\right )\in \left (T\left (A\right )\times B\right )\mapsto \left (\left (I_1, b\right ), I_2\right )\in T\left (A\times B\right )$ qui est continue par composition, couplage, projections.

$s\left (\eta_A\left (a\right ), b\right ) = s\left (\sigma \mapsto \left (a, \sigma\right ), b\right ) = \sigma \mapsto \left (\left (a, b\right ), \sigma\right ) = \eta_{A \times B}\left (a, b\right )$.





On a ainsi deux applications ``naturelles" $s_1$ et $s_2$ de $T\left (A\right ) \times T\left (B\right ) \rightarrow T\left (A \times B\right )$ 

avec $\forall \left (I, J\right ) \in T\left (A\right ) \times T\left (B\right ),\  $ $s_1\left (I, J\right ) = \sigma \mapsto \left (\left (I_1\left (\sigma\right ), J_1\left (\sigma\right )\right ), I_2\left (J_2\left (\sigma\right )\right )\right )$ et $s_2\left (I, J\right ) = \sigma \mapsto \left (\left (I_1\left (\sigma\right ), J_1\left (\sigma\right )\right ), J_2\left (I_2\left (\sigma\right )\right )\right )$. Elles sont de même continues.


Avec A=B=1, en simplifiant on écrit $s1:\left (I,J\right )\mapsto I \circ J$ $s1:\left (I,J\right )\mapsto J \circ I$ donc elles sont différentes et  $sq = s_2$

\subsection*{Question 8}

On définie la fonction $C \mapsto \Cr{C}_A$ par induction sur la structure de $C$.

On utilise la fonction $Cond : \left (\M \rightarrow \{ true, false \}\right ) \times P\left (A\right ) \times P\left (A\right )\rightarrow P\left (A\right )$ adapté de celle utilisé à la question 4. Elle est continue pour les mêmes raisons.

\begin{itemize}
    \item $C = \texttt{skip}$
    
    On définit $\Cr{C}_A = \sigma \mapsto \left (\bot_A, \sigma\right ) \in T\left (A\right )$. On a $\Co{C, \sigma} \Downarrow \sigma'$ si et seulement si $\exists a \Cr{C}_A\left (\sigma\right ) = \left (a, \sigma'\right )$.
    
    \item $C = C' ; C$
    
    Soit $\sigma$, on pose $ \left (\left (a_1,a_2\right ),S\right )=sq\left ([\Cr{C''}_A,\Cr{C'}_A\right )\left (\sigma\right )$ 
    
    On définit alors $\Cr{C}_A\left (\sigma\right ) =\left ( \left (\comp\left (a_1,a_2\right )\right ), S\right )$. On a bien $\Cr{C}_A \in T\left (A\right )$ par composition. 
    On a $\Co{C, \sigma} \Downarrow \sigma'$ si et seulement si  $\exists \sigma'' \: \exists a, a' \in A$ $\Cr{C'}_A\left (\sigma\right ) = \left (a, \sigma''\right )$ et $\Cr{C''}_A\left (\sigma''\right ) = \left (a', \sigma'\right )$ c'est à dire $\exists a \in A \: \Cr{C''}_A\left ({\Cr{C'}_A}_1\left (\sigma\right )\right ) = \left (a, \sigma'\right )$ ce que l'on voulait.
    
    \item $C = \texttt{r := a}$
    
     On définit $\Cr{C}_A = \sigma \mapsto w_r\left (\mathcal{A}[e]\right )$ qui est continue et vérifie la propriété par hypothèse.
    
    \item $C = \texttt{if } b \texttt{ then } C_1 \texttt{ else } C_2$

        On définit $\Cr{C} = Cond\left (\mathcal{B}\left (b\right ), C_1, C_2\right )$. Idem à la question 4.
    
    \item $C = \texttt{while } b \texttt{ do } C'$
    
    On définit  $iter\left (f\right ) = \sigma \mapsto \left (\comp\left (sq\left (\Cr{C'}, f\right )_1\left (\sigma\right )\right ), seq\left (\Cr{C'}, f\right )_2\left (\sigma\right )\right )$.
     
     On prend $\Cr{C}_A = F\left (f \mapsto Cond\left (\mathcal{B}\left (b\right ), iter\left (f\right ), Id\right )\right )$ (Cf question 3). La définition est valide car $f \mapsto Cond\left (\mathcal{B}\left (b\right ), iter\left (f\right ), Id_\M\right )$ est continue comme composé de fonctions continue. Donc $\Cr{C}_A$ convient comme à la question 4.
\end{itemize}

\subsection*{Question 9}

Ici on définit $\comp$ par l'addition avec $\bot_\N$ interprété comme 0. On fixe $w_r(n)\rightarrow (1,\Co{n/r})$

Ainsi si $\Cr{C}(\sigma)=(n,\sigma')$ l'exécution de $C$ a opéré n fois la commande $\texttt{r := a}$, qui correspond au nombre d'écriture mémoire.

\subsection*{Question 10}

On peut se servir de $A$ pour stocker un booléen indiquant s'il y a eu un \texttt{abort}. $A$ devient $A\times \bool$.

\begin{itemize}
\item On définit alors  $[[\texttt{abort}]](\sigma)=((\bot_A,tt),\sigma)$

\item Pour la séquence $C = C' ; C$
    
    Soit $\sigma$, on pose $ \left (\left ((a_1,b_1),(a_2,b2)\right ),S\right )=sq\left ([\Cr{C''}_A,\Cr{C'}_A\right )\left (\sigma\right )$ 
    
    On définit alors $\Cr{C}_A\left (\sigma\right ) =\left ( \left (\comp\left (a_1,a_2\right ),ff\right ), S\right )$ si $b_1=b_2=ff$ ou $\left ( \left ( a_1 ,tt\right ), S\right )$ sinon .
    
    \item Pour les boucles on procède similairement.

\end{itemize}

Ainsi l'état mémoire renvoyé est celui juste après le premier \texttt{abort} rencontré.

\end{document}
